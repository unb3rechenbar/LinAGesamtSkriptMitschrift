\documentclass[../../main.tex]{subfiles}
\begin{document}
\section{Infima und Suprema, Minima und Maxima}

\begin{df}\label{12.1.1}
		Sei $A$ eine Menge. Eine \emph{Halbordnung}\index{Halbordnung@{\bf Halbordnung}} (auch: \emph{partielle Ordnung}) ist eine Relation [$\to$ \ref{1.3.1} (e)] $\preceq$ auf $A$, für die gilt:
	\begin{itemize}
		\item[(R)] $\forall a\in A: a\preceq a$ - \enquote{reflexiv},
		\item[(A)] $\forall a,b\in A: (a\preceq b)\land (b\preceq a)\implies a=b$ - \enquote{antisymmetrisch},
		\item[(T)] $\forall a,b,c\in A: (a\preceq b)\land (b\preceq c)\implies a\preceq c$ - \enquote{transitiv}.
	\end{itemize}
	Gilt zusätzlich
	\begin{itemize}
		\item[(L)] $\forall a,b\in A: (a\preceq b)\lor (b\preceq a)$ - \enquote{linear},
	\end{itemize}
	so heißt $\preceq$ sogar eine \emph{Ordnung}\index{Halbordnung@{\bf Halbordnung}!Ordnung/ lineare Ordnung} (auch: \textit{lineare Ordnung}, \emph{Totalordnung}).
\end{df}

\begin{bsp}\label{12.1.2}
	\begin{enumerate}[\normalfont(a)]
		\item Sei $A$ eine Menge von Mengen. Dann wird durch
		\begin{align*}
			\forall N,M\in A: N\preceq M:\Longleftrightarrow N\subseteq M
		\end{align*}
		per Mengeninklusion eine Halbordnung auf $A$ definiert.
		\item Die Teilsbarkeitsbeziehung auf $\N$ definiert durch
		\begin{align*}
			a\mid b:\Longleftrightarrow \exists c\in \N: b=ca.
		\end{align*}
		ist eine Halbordnung auf den natürlichen Zahlen.
		\item Die \emph{natürliche Ordnung}\index{Halbordnung@{\bf Halbordnung}!Ordnung/ lineare Ordnung! natürliche Ordnung} $\le$ auf $\N$ ist eine Ordnung.
		\item Die rationale Ordnung $\le$ auf $\{\infty\}\cup\R\cup\{-\infty\}$ ist eine Ordnung.
	\end{enumerate}
\end{bsp}

\begin{bem}\label{12.1.3}
	\begin{enumerate}[\normalfont(a)]
		\item Ist $\preceq$ eine (Halb-)Ordnung auf $A$, dann auch $\succeq$ vermöge der Relation $\forall a,b\in A: a\succeq b:\Longleftrightarrow b\preceq a$. Es ist $\succeq$ die zu $\preceq$ \emph{inverse} (Halb-)Ordnung.
		\item Ist $\preceq$ eine (Halb-)Ordnung auf $A$, so finden wir auch eine Halbordnung für ein $B\subseteq A$ über $\preceq':=\preceq\cap (B\times B)$. Es ist $\preceq'$ die \emph{Einschränkung} von $\preceq$ auf $B$.
	\end{enumerate}
\end{bem}

\begin{df}\label{12.1.4}
	Ein geordnetes Paar $(A,\preceq)$ bestehend aus einer Menge $A$ und einer (Halb-)Ordnung $\preceq$ auf $A$ heißt (halb-)geordnete Menge. Wir nennen $A$ die zugrundeliegende Menge (auch: Trägermenge/ Universum) und $\preceq$ die (Halb-)Ordnung von $(A,\preceq)$.
\end{df}

\begin{bem}\label{12.1.5} 
	Wie immer pflegen wir einen \enquote{schlamigen} Sprachgebrauch, z.B. \enquote{Sei $A$ eine geordnete Menge und seien $a,b\in A$ mit $a\preceq b$}.
\end{bem}

\begin{df}\label{12.1.6}
	Sei $(A,\preceq)$ eine halbgeordnete Menge und $B\subseteq A$. Ein Elemente $a\in A$ heißt \case{\emph{obere}\index{Halbordnung@{\bf Halbordnung}!obere Schranke}}{\emph{untere}\index{Halbordnung@{\bf Halbordnung}!untere Schranke}} Schranke von $B$, wenn $\forall b\in B:$ \case{$a\preceq b$}{$b\preceq a$} gilt. Wir notieren
	\begin{itemize}
		\item $\ub(B):=\{a\in A\mid a\text{ obere Schranke}\}$
		\item $\lb(B):=\{a\in A\mid a\text{ untere Schranke}\}$.
	\end{itemize}

	\noindent Ein Element $a\in A$ heißt \case{\emph{Minimum}\index{Halbordnung@{\bf Halbordnung}!Minimum} (auch: kleinstes Element)}{\emph{Maximum}\index{Halbordnung@{\bf Halbordnung}!Maximum} (auch: größtes Element)}, wenn \case{$a\in B\cap \lb(B)$}{$a\in B\cap \ub(B)$} gilt.  [Wenn existent, ist ein solches Element wegen der Antisymmetrie von $\preceq$ eindeutig und wir notieren dieses mit \case{$\min B$}{$\max B$}.]\\
			
	\noindent Ein Element $a\in A$ heißt \case{\emph{Infimum}\index{Halbordnung@{\bf Halbordnung}!Infimum} (auch: größte untere Schranke)}{\emph{Supremum}\index{Halbordnung@{\bf Halbordnung}!Supremum} (auch: kleinste obere Schranke)} von $B$, wenn \case{$a=\max\lb(B)$}{$a=\min\ub(B)$} gilt. Wenn existent, notieren wir ein solches Element mit $\left\{\begin{matrix*}\inf B \\ \sup B\end{matrix*}\right\}$.
\end{df}

\begin{pro}\label{12.1.7}
	Sei $A$ eine halbgeordnete Menge und $B\subseteq A$.
	\begin{enumerate}[\normalfont(a)]
		\item Besitzt $B$ ein \case{Minimum}{Maximum}, so auch ein \case{Infimum}{Supremum} und es gilt \case{$\inf B=\min B$}{$\sup B=\max B$}.
		\item Besitzt $B$ ein \case{Infimum}{Supremum}, so ist dieses genau dann das \case{Minimum}{Maximum}, wenn dieses in $B$ liegt.
		\item Sei $a\in A$. Dann ist $a$ genau dann \case{Infimum}{Supremum} von $B$, wenn gilt $\forall c\in A:$ \case{$c\in \lb(B)\Longleftrightarrow c\preceq a$}{$c\in \ub(B)\Longleftrightarrow a\preceq c$}.
	\end{enumerate}
\end{pro}
\begin{cproof} 
	Wir zeigen die Aussagen je für das Infimum/ Minimum. Der entsprechende Beweis für das Supremum/ Maximum folgt dann über die Halbordnung $\succeq$.
	\begin{itemize}
		\item[(a)] Sei $m:=\min B$. Dann gilt $\forall b\in B: m\preceq b$ und wegen $m\in B$ auch $\forall l\in \lb(B): l\preceq b$. Insgesamt folgt $b=\max\lb(B)=\inf B$.
		\item[(b)] Sei $i:= \inf B$. Ist $i$ das Minimum von $B$, liegt dieses per Definition bereits darin. Liegt $i$ in $B$, dann folgt aus $i\in\lb(B)$ die gewünschte Minimaleigenschaft.
		\item[(c)] Sei $a\in A$. Es gilt
		\begin{align*}
			(a=\inf A)&\Longleftrightarrow (a\in\max \lb(B))\\
			&\Longleftrightarrow (\forall c\in A: c\in\lb(B)\Rightarrow c\preceq b)\land a\in \lb(B).
		\end{align*}
		Die letzte Aussage ist erkennbar äquivalent zu der Behauptung, da $a\preceq a$.
	\end{itemize}
\end{cproof}

\begin{bsp}\label{12.1.8}
	$\;$
	\begin{enumerate}[\normalfont(a)]
		\item Sei $M$ eine Menge und $A:=\Pow(M)$ durch Inklusion halbgeordnet. Dann gilt für $B\subseteq A$:
		\begin{itemize}
			\item $\inf B:=\begin{cases} M & \text{wenn } B=\emptyset\\ \bigcap B & \text{sonst}\end{cases}$
			\item $\sup B:=\bigcup B$
		\end{itemize}
		\item Sei $V$ ein $K$-Vektorraum und
		\begin{align*}
			A:=\{U\mid U \text{ Unterraum von } V\}
		\end{align*}
		durch Inklusion halbgeordnet. Dann gilt für $B\subseteq A$:
		\begin{itemize}
			\item $\inf B:=\begin{cases} V & \text{wenn }B=\emptyset\\ \bigcap B & \text{sonst}\end{cases}$.
			\item $\sup B:=\lin\left(\bigcup B\right)$
		\end{itemize}
		\item Sei
		\begin{align*}
			A:=\{\underbrace{\{1\}}_{=a},\underbrace{\{2\}}_{=b},\underbrace{\{3\}}_{=c},\underbrace{\{1,2\}}_{=d},\underbrace{\{2,3\}}_{=e}\}
		\end{align*}
		durch Inklusion halbgeordnet. Es gilt:
		\begin{table}[H]
			\centering
			\begin{tabular}{c|c|c|c|c|c|c}
				$B$ & $\lb(B)$ & $\ub(B)$ & $\inf(B)$ & $\min(B)$ & $\sup(B)$ & $\max(B)$\\
				\hline
				$\emptyset$ & $A$ & $A$ & $-$ & $-$ & $-$ & $-$\\
				$\{a\}$ & $\{a\}$ & $\{a,d\}$ & $a$ & $a$ & $a$ & $a$\\
				$\{b\}$ & $\{b\}$ & $\{b,d,e\}$ & $b$ & $b$ & $b$ & $b$\\
				$\{d\}$ & $\{a,b,d\}$ & $\{d\}$ & $e$ & $e$ & $e$ & $e$\\
				$\{a,b\}$ & $-$ & $\{d\}$ & $-$ & $-$ & $d$ & $-$\\
				$\{d,e\}$ & $-$ & $\{b\}$ & $b$ & $-$ & $-$ & $-$
			\end{tabular}
		\end{table}
		\item In $(\N, \mid)$ gilt für $B\subseteq \N$:
		\begin{itemize}
			\item $\lb(B)$ ist die Menge der gemeinsamen Teiler $B$.
			\item $\ub(B)$ ist die Menge der gemeinsamen Vielfachen von $V$.
			\item $\inf(\emptyset)$ existiert nicht wegen der Unbeschränktheit von $\lb(\emptyset)=\N$.
		\end{itemize}
		Für $B\neq \emptyset$ gilt weiter:
		\begin{itemize}
			\item $\inf(B)=\underbrace{\text{größter}}_{\text{in Teilerrelation}} \text{ gemeinsamer Teiler aller Zahlen in }B$
			\item $\sup(B)=\underbrace{\text{kleinste}}_{\text{in Teilerrelation}} \text{ gemeinsame Vielfache aller Zahlen in }B\text{, wenn existent!}$
		\end{itemize}		
		\item In $\N$ mit der natürlichen Ordnung gilt:
		\begin{itemize}
			\item $\sup\emptyset=1$, aber das Maximum von $\emptyset$ existiert nicht. Ebenso existieren das Infimum und Minimum von $\emptyset$ nicht.
			\item ist $B\neq\emptyset$, dann hat $B$ immer ein Minimum und somit Infimum. Die Begriffe von Maximum und Supremum gleichen sich hier. Also
			\begin{align*}
				B\text{ hat Maximum}\Longleftrightarrow B\text{ hat Minimum}\Longleftrightarrow B\text{ ist endlich}.
			\end{align*}
		\end{itemize}
		\item In der auf natürliche Weise geordneten Menge $\R_\infty:=\{-\infty\}\cup\R\cup\{\infty\}$ gilt: per Supremumsaxiom in $\R$ und hinzugefügten Elementen zu $\R_\infty$ hat jede Teilmenge $B\subseteq\R_\infty$ ein Infimum und Supremum. Insbesondere gilt für das Infimum:
		\begin{itemize}
			\item $\inf B=-\infty\Longleftrightarrow B\text{ hat keine reelle untere Schranke}$
			\item $\inf B=\infty\Longleftrightarrow B\in\{\emptyset,\{\infty\}\}$
		\end{itemize}
		Anaolog gilt für das Supremum:
		\begin{itemize}
			\item $\sup B=\infty\Longleftrightarrow B\text{ hat keine reelle obere Schranke}$
			\item $\sup B=-\infty\Longleftrightarrow B\in\{\emptyset,\{-\infty\}\}$
		\end{itemize}
	\end{enumerate}
\end{bsp}

\section{Das Zornsche Lemma}

\begin{nt}\label{12.2.1} 
	Wird eine Halbordnung $R$ auf $A$ durch $\preceq$, $\le$, $\succeq$ oder $\ge$ notiert, so bezeichnen man mit $\prec$, $<$, $\succ$ oder $>$ die Relation $R'$ auf $A$ definiert durch
	\begin{align*}
		\forall a,b\in A: aR'b:\Longleftrightarrow (aRb\land a\neq b)
	\end{align*}
\end{nt}	

\begin{df}\label{12.2.2} 
	Sei $(A\preceq)$ eine halbgeordnete Menge und $B\subseteq A$. Mann nennt $b$ ein \case{\emph{maximales}\index{Halbordnung@{\bf Halbordnung}!maximales Element}}{\emph{minimales}\index{Halbordnung@{\bf Halbordnung}!minimales Element}}  Element von $B$, wenn $b\in B$ und \case{$\nexists c\in B: b\prec c$}{$\nexists c\in B: c\prec b$} gilt.
\end{df}

\begin{bem}\label{12.2.3} 
	Dies ist nicht zu verwechseln mit den in \ref{12.1.6} eingeführten Begriffen. Alle Maxima und Minima sind zwar auch maximale bzw. minimale Elemente, die Umkehrung dieser Aussage gilt jedoch im Allgemeinen \underline{nicht}. In z.B. $(\N,\mid)$ ist $3$ ein maximales Element von $\{3,28\}$, aber kein Maximum. Diese Menge besitzt kein Maximum.
\end{bem}
		
\begin{df}\label{12.2.4}
	Sei $(A,\preceq)$ eine halbgeordnete Menge und $C\subseteq A$. Dann heißt $C$ eine \emph{Kette}\index{Halbordnung@{\bf Halbordnung}!Kette} (in $(A,\preceq)$), wenn die Einschränkung von $\preceq$ auf $C$ eine Ordnung ist, d.h. zusätzlich gilt:
	\begin{itemize}
		\item $\forall b,c\in C: (b\preceq c)\lor(c\preceq b)$ - \enquote{Totalität}
	\end{itemize}
	Anschaulich wird $C$ durch $\preceq\vert_{C\times C}$ linear zu einer \enquote{Kette}.
\end{df}
		
\begin{bsp}\label{12.2.5}
	\begin{enumerate}[\normalfont(a)]
		\item Es ist $B:=\{\emptyset,\{2\},\{1,2,4\}\}$ eine Kette in $(\Pow(\{1,2,3,4\}),\subseteq)$.
		\item Die Menge $\{2^n\mid n\in\N\}$ ist eine Kette in $(\N,\mid)$, aber nicht $\{2,3\}$ denn $2\nmid 3$ und $3\nmid 2$.
	\end{enumerate}
\end{bsp}

\begin{lem}\label{12.2.6} 
	Sei $M$ eine Menge und $\Pow(M)$ durch Inklusion halbgeordnet. Sei $B\in\Pow(M)$ derart, daß für jede Kette $C\subseteq B$ gilt $\bigcup C\in B$. Es ist $B$ gewissermaßen abgeschlossen unter Supremumsbildung. Ferner sei $f: B\rightarrow B$ eine Abbildung mit $\forall N\in B: N\subseteq f(N)$. Dann gibt es $N_0\in B$ mit $N_0=f(N_0)$.
\end{lem}
\begin{cproof}
	Wir nennen $B'\subseteq B$ als $f$-induktiv, wenn $\forall N\in B': f(N)\in B'$ und $\bigcup C\in B'$ für jede Kette $C\subseteq B'$ gilt. Somit ist $B$ per Voraussetzung $f$-induktiv.\\	
	\schritt{1}
	\begin{behbox}
		\hbeh{}
			Man darf annehmen, daß keine echte Teilmenge von $B$ $f$-induktiv ist. \\
		\hbeg{}
			Wir zeigen, daß eine kleinste $f$-induktive Teilmenge von $B$ existiert (\enquote{minimalste} würde aber reichen!). Setze
			\begin{align*}
				B'':=\bigcap\{B'\subseteq B\mid B'\text{ ist}f\text{-induktiv}\}.
			\end{align*}
			Klarerweise ist $B''$ dann $f$-induktiv. Wir könnten dann mit $B''$ statt $B$ arbeiten - da aber $B''\subseteq B$, folgt die Aussage somit auch für $B$. \oe sei also $B=B''$.
	\end{behbox}
	\noindent Wir definieren die Menge der Fixpunkte als  solche, für welche alle Teilmengen auch dessen $f$-Vergrößerung kleiner als diese ist:
	\begin{align*}
		B':=\{N'\in B\mid \forall N\in B: (N\subset N'\implies f(N)\subseteq N')\}.
	\end{align*}
	Für alle $N'\in B'$ definieren wir zudem
	\begin{align*}
		B'_N:=\{N\in B\mid(N\subseteq N')\lor(f(N')\subseteq N)\}.
	\end{align*}
	Dies ist also die Menge aller Punkte unter $N'$ (in der Halbordnung auf $\Pow(M)$) und alle Punkte, welcher größer als die $f$-Vergrößerung von $N'$ ist. Es fehlen alle Punkte echt zwischen $N'$ und $f(N')$.
	\schritt{2}
	\begin{behbox}
		\hbeh{}
			Für jedes $N'\in B'$ ist $B'_N$ $f$-induktiv. \\
		\hbeg{}
			Sei $N'\in B'$. Zu zeigen sind
			\begin{enumerate}[\normalfont(b)]
				\item $\forall N\in B'_N: f(N)\in B_N'$,
				\item Für jede Kette $C\subseteq B_N'$ gilt $\bigcup C\in B_N'$.
			\end{enumerate}
			Wir zeigen die Behauptungen nacheinander:
			\begin{enumerate}[\normalfont(b)]
				\item Sei $N\in B'_N$. Dann $N\in B$ und $N\subseteq N'$ oder $f(N')\subseteq N$. Zu zeigen ist $f(N)\in B_N'$. $f(N)\in B$ gilt immer. Ist $N\subseteq N'$, folgt entweder $N=N'$ oder $N\subset N'$. Ersterer ist klar wegen $N'\subseteq N'$. Aus letzterem folgt $f(N)\subseteq N'$, da $N'$ ein Fixpunkt ist. Ist $f(N')\subseteq N$, dann folgt wegen $N\subseteq f(N)$ auch $f(N')\subseteq f(N)$.
				\item Sei $C\subseteq B_N'$ eine Kette. Zu zeigen ist $\bigcup C\in B_N'$. Es gilt immer $\bigcup C\in B$. Zu zeigen ist $\bigcup C\subseteq N'$ oder $f(N')\subseteq \bigcup C$. Gelte ersteres nicht, zu zeigen ist dann letzteres. Wir können $N\in C$ mit $N\subsetneq N'$ wählen. Mit $N\in B'_N$ folgt $f(N')\subseteq N\subseteq \bigcup C$.
			\end{enumerate}
	\end{behbox}
	Insgesamt folgt für alle $N'\in B'$, daß $B'_N=N$ gilt. Also ist
	\begin{align*}
		(*)\ \forall N'\in B': \forall N\in B: (N\subseteq N'\lor f(N')\subseteq N').
	\end{align*}
	\schritt{3}
	\begin{behbox}
		\hbeh{}
			$B'$ ist $f$-induktiv. \\
		\hbeg{}
			Zu zeigen sind 
			\begin{enumerate}[\normalfont(a)]
				\item $\forall N\in B': f(N)\in B'$,
				\item Für jede Kette $C\subseteq B'$ gilt $\bigcup C\in B'$.
			\end{enumerate}
			Auch diese Aussagen zeigen wir nacheinander:
			\begin{enumerate}[\normalfont(a)]
				\item Sei $N'\subseteq B'$. Zu zeigen ist $f(N')\in B$ und $\forall N\in B: (N\subset f(N')\implies f(N)\subseteq f(N'))$. Ersteres ist klar. Sei nun $N\in B$ mit $N\subset f(N')$. Zu zeigen ist $f(N)\subseteq f(N')$. Wegen (*) folgt aber $N\subseteq N'$. Sei $\OE$ $N\neq N'$. Dann $f(N)\subseteq N'\subseteq f(N')$, da $N'$ ein Fixpunkt ist.
				\item Sei $C\subseteq B'$ eine Kette. Zu zeigen ist $\bigcup C\in B$ und $\forall N\in B: (N\subset \bigcup C)\implies f(N)\subseteq \bigcup C)$. Ersteres ist klar. Sei nun $N\in B$ mit $N\subset \bigcup C$. Zu zeigen ist $f(N)\subseteq \bigcup C$. Wegen $\bigcup C\subsetneq N$ gibt es $N'\in C\subseteq B'$ mit $N'\subsetneq N$. Insbesondere ist $f(N')\subseteq N$. Wegen (*) folgt $N\subseteq N'$ und hier sogar $N\subset N'$. Also $f(N)\subseteq N'\subseteq \bigcup C$ und wir sind fertig.
			\end{enumerate}
	\end{behbox}
	\noindent Wegen \textit{Schritt 1} ist $B'=B$, weshalb aus (*) folgt, daß $B$ eine Kette ist. Da B $f$-induktiv ist, folgt $N_0:=\bigcup B\in B$ und $f(N_0)\in B$. Daraus folgt $N_0\subseteq f(N_0)\subseteq \bigcup B=N_0$, also $N_0=f(N_0)$.
\end{cproof}

\begin{lem}\label{12.2.7}
	Sei $A$ eine halbgeordnete Menge. Dann gibt es eine maximale Kette $C$ in $A$.
\end{lem}
\begin{cproof}
	Bezeichne $\mathcal{C}$ die durch Inklusion halbgeordnete Menge aller Ketten in $A$:
	\begin{align*}
		\mathcal{C}:=\{C\subseteq A\mid C\text{ ist Kette}\}.
	\end{align*}
	Nehmen wir an, es gäbe kein maximales Element in $\mathcal{C}$, also zu jedem $C\in\mathcal{C}$ gibt es ein $f(C)\in\mathcal{C}$ mit $C\subset f(C)$. Dies definiert eine Abbildung $f: \mathcal{C}\rightarrow \mathcal{C}$. Wir zeigen nun, für jede Kette $K\subseteq\mathcal{C}$ gilt $\bigcup K\in \mathcal{C}$. Sei also $K\subseteq\mathcal{C}$ eine Kette und seien $a,b\in \bigcup K$. Zu zeigen ist $a\preceq b\lor b\preceq a$. Wählen wir nun $C_1,C_2\in K$ so, daß $a\in C_1$ und $b\in C_2$. Da $K$ eine Kette ist, folgt $(C_1\subseteq C_2)\lor (C_2\subseteq C_1)$. $\OE$ gelte ersteres. Dann $a,b\in C_2$ und da $C_2$ eine Kette ist, folgt die Behauptung.\\
				
	Insgesamt ist dann $\mathcal{C}$ $f$-induktiv. Mit Lemma \eqref{12.2.6} finden wir dann aber eine Kette $C_0\in\mathcal{C}$ mit $C_0=f(C_0)$, was unserer Annahme widerspricht. Per Widerspruch sind wir somit fertig.
\end{cproof}

\begin{sat}\label{12.2.8}
	Als \emph{Lemma von Zorn}\index{Halbordnung@{\bf Halbordnung}!Zorn'sche Lemma} $\left[\begin{matrix*}\text{Max August Zorn}\\ \text{*1906,\dag 1993}\end{matrix*}\right]$ wird der folgende Satz von Kuratowski $\left[\begin{matrix*}\text{Kazimierz, Kuratowski}\\ \text{*1896,\dag 1980}\end{matrix*}\right]$ bezeichnet.\\
	Sei $A$ eine halbgeordnete Menge derart, daß jede Kette in $A$ eine obere Schranke in $A$ besitzt. Dann besitzt $A$ ein maximales Element.
\end{sat}
\begin{cproof}
	Wähle nach Lemma \eqref{12.2.7} eine maximale Kette $C$ in A. Wähle $a\in\ub(C)$ gemäß Voraussetzung. Wir behaupten, daß $a$ ein maximales Element von $A$ ist. Sei hierfür $c\in A$ mit $a\preceq c$. Zu zeigen ist $a=c$. Es ist wohl dann $C\cup\{a,c\}$ eine Kette in $A$. Da $C$ maximal ist, folgt aber $a,c\in C$. Da $a\in\ub(C)$, muss $c\preceq a$ gelten und per Antisymmetrie also $a=c$.
\end{cproof}

\begin{bem}\label{12.2.9}
	Daß die leere Kette in einer halbgeordneten Menge $A$ eine obere Schranke besitzt, führt dazu, daß $A$ nichtleer sein darf. Deshalb formulieren manche Autoren: \enquote{Sei $A$ eine nichtleere halgeordnete Menge derart, daß jede nichtleere Kette \ldots}.
\end{bem}

\begin{kor}\label{12.2.10}
	Sei $A$ eine halbgeordnete Menge derart, daß jede nichtleere Kette in $A$ eine obere Schranke in $A$ besitzt. Dann gibt es zu jedem $a\in A$ ein maximales Element $b\in A$ mit $a\preceq b$.
\end{kor}
\begin{cproof}
	Betrachte $A':=\ub(\{a\})$ mit der auf $A'$ eingeschränkten Halbordnung. Wende das Zornsche Lemma auf $A'$ an und erhalte ein maximales Element $b$ von $A'$ zu erhalten. Es ist $b$ auch ein maximales Element von $A$, denn ist $c\in A$ mit $b\preceq c$, so gilt auch $c\in A'$. Also folgt $b=c$.
\end{cproof}

\begin{kor}\label{12.2.11}
	Sei $A$ eine durch Inklusion halbgeordnete Menge von Mengen derart, daß für jede nichtleere Kette $C\subseteq A$ gilt \case{$\bigcup C\in A$}{$\bigcap C\in A$}. Dann gibt es zu jedem $M\in A$ ein \case{maximales}{minimales} $N\in A$ mit \case{$M\subseteq N$}{$ N\subseteq M$}.
\end{kor}

\begin{bem}\label{12.2.12}
	Wie in \ref{1.1.2} angedeutet, gibt es bei der Mengenbildung Spielregeln einzuhalten. Dies wird auch durch obiges Korollar deutlich, welches im Widerspruch zur Existenz der Menge aller Mengen steht. Denn diese hat keine maximalen Elemente.
\end{bem}

\begin{ans}
	Das Zorn'sche Lemma besagt, daß Mengen nicht furchtbar groß sein können, was folgender Gedanke verdeutlicht: Sei $A$ eine Menge, die alle Voraussetzung für das Lemma von Zorn erfüllt. Wie in Bemerkung \ref{12.2.9} erläutert, muss es ein Element $a_1\in A$ geben. Ist dieses nicht maximal, finden wir $a_2,a_3\ldots $, die größer als $a_1$ sind und erhalten die Kette
	\begin{align*}
		a_1,a_2,a_3,\ldots 
	\end{align*}
	Im schlimmsten Fall ist diese unendlich. Da aber jede Kette nach Voraussetzung ein maximales Element hat, gibt es ein $b_1$ in $A$, das größer als $a_1,a_2,a_3,\ldots $ ist und weiter eine Kette $b_1,b_2,b_3,\ldots $. Irgendwann erhält man so im schlimmsten Fall unendlich viele Ketten
	\begin{align*}
		&a_1,a_2,a_3,\ldots \\
		&b_1,b_2,b_3,\ldots \\
		&c_1,c_2,c_3,\ldots \\
		&\ldots ,
	\end{align*}
	wobei sichtbar $a_1,b_1,c_1,\ldots $ wieder eine Kette ist, mit welcher man gleiche Überlegungen wie zu der Kette $a_1,a_2,a_3,\ldots $ anstellen kann. Nach dem Zorn'schen Lemma muss aber irgendwann damit Schluss sein - es muss ein maximales Element geben.
\end{ans}

\section{Existenz von Basen in beliebigen Vektorräumen}
\begin{sat}\label{12.3.1}
	[$\to$ \ref{6.2.14}] 
	Sei $V$ ein Vektorraum, $F\subseteq G\subseteq V$. Sei ferner $F$ linear unabhängig in $V$ und $G$ ein 	Erzeugendensystem von $V$. Betrachte die durch Inklusion halbgeordnete Menge
	\begin{align*}
		\mathcal{A}:=\{C\mid F\subseteq C\subseteq G\}.
	\end{align*}
	Dann sind äquivalent folgende Aussagen für alle $F\subseteq B\subseteq G$:
	\begin{enumerate}[\normalfont (a)]
		\item $B$ ist eine Basis von $V$,
		\item $B$ ist ein minimales Element von
		\begin{align*}
			\{C\in \mathcal{A}\mid C\in\text{ Erzeugendensystem von } V\},
		\end{align*}
		\item $B$ ist ein maximales Element von
		\begin{align*}
			\{C\in \mathcal{A}\mid C\in\text{ linear unabhängig in } V\}.
		\end{align*}
	\end{enumerate}
\end{sat}
\begin{cproof}
	Schon gezeigt.
\end{cproof}

\begin{kor}\label{12.3.2}
	Sei $V$ ein Vektorraum, $F\subseteq G\subseteq V$. Sei ferner $F$ linear unabhängig in $V$ und $G$ ein Erzeugendensystem von $V$. Dann gibt es eine Basis $B$ von V mit $F\subseteq B\subseteq G$.
\end{kor}
\begin{cproof}
	Betrachte die durch Inklusion halbgeordnete Menge
	\begin{align*}
		\mathcal{B}:=\{C\mid F\subseteq C\subseteq G, C\text{ linear unabhängig in }V\}.
	\end{align*}
	Gemäß \ref{12.3.1}(c) ist zu zeigen, daß $\mathcal{B}$ ein maximales Element besitzt. Nach dem Zornschen Lemma reicht es hierfür zu zeigen, daß jede Kette in $\mathcal{B}$ eine obere Schranke besitzt. Sei also $K\subseteq \mathcal{B}$ eine Kette. Falls $K=\emptyset$, so ist $F$ eine obere Schranke von $K$. Sei also $K\neq \emptyset$. Wir behaupten, daß
	\begin{align*}
		C:=\bigcup K
	\end{align*}
	eine obere Schranke von $K$ ist. Es ist $F\subseteq C\subseteq G$ klar. Zu zeigen bleibt, daß $C$ linear unabhängig ist. Hierfür zeigen wir, daß jede endliche Teilmenge von $C$ linear unabhängig ist. Sei also $C'\subseteq C$ endlich. Da $C'\subseteq K$, gibt es eine endliche Teilkette $K'\subseteq K$ mit $C'\subseteq \bigcup K'$. $\OE$ gelte $K'\neq \emptyset$. Da $K'$ eine endliche nichtleere Kette ist, hat sie ein maximales Element $m_k:=\bigcup K'$. Daher gilt gilt $\bigcup K'\in K'\subseteq K\subseteq \mathcal{B}$. Insbesondere ist $K'$ linear unabhängig und damit auch $C'$.
\end{cproof}
\end{document}